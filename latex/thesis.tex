\documentclass[a4paper, 11pt]{report}

\usepackage[hidelinks]{hyperref}
\usepackage{amssymb}
\usepackage{amsmath}
\usepackage{amsthm}
\usepackage[utf8]{inputenc}
\usepackage[T1]{fontenc}
\usepackage{lmodern}
\usepackage{lipsum}
\usepackage[onehalfspacing]{setspace}
\usepackage[top=30mm, left=25mm, right=25mm, bottom=20mm]{geometry}
\usepackage{graphicx}
\usepackage{nameref}
\usepackage{graphicx}
\usepackage{xcolor}
\usepackage{colortbl}
\usepackage{subcaption}
\usepackage{cleveref}
\usepackage{tikz}
\usepackage{float}
\usepackage{listings}
\usepackage{extarrows}
\usepackage[main=english]{babel}
\usepackage{marvosym}
\usepackage[nottoc]{tocbibind}
\hypersetup{hypertexnames=false}

\lstset{numbers=left, numberstyle=\tiny, numbersep=5pt}
\lstset{language=Python}

\usetikzlibrary{matrix}

\DeclareMathOperator{\rel}{\sim_R}
\renewcommand{\emph}[1]{\textit{#1}}
\newcommand{\mytitle}[1]{\LARGE{#1}\normalsize\\[0.3em]}
\newcommand{\titlespace}{\vspace{2em}}
\newcommand{\smallspace}{\vspace{1em}}
\newcommand{\hugespace}{\vspace{17em}}
\newcommand{\logoheight}{4em}

\theoremstyle{definition}
\newtheorem{definition}{Definition}[section]
\newtheorem{example}[definition]{Example}
\newtheorem{theorem}[definition]{Theorem}
\newtheorem{corollary}[definition]{Corollary}
\newtheorem*{remark}{Remark}
\newenvironment{myAbstract}{\section*{Abstract}}{}

\begin{document}
\pagenumbering{gobble}
\newgeometry{top=35mm, left=20mm, right=20mm, bottom=10mm}
\begin{titlepage}
	\begin{center}
		\begin{minipage}{.49\textwidth}
			\flushleft
			\includegraphics[height=\logoheight]{../assets/formal/logo_gau.png}
		\end{minipage}
		\begin{minipage}{.49\textwidth}
			\flushright
			\includegraphics[height=\logoheight]{../assets/formal/logo_dlr.png}	
		\end{minipage}
		\begin{minipage}{.49\textwidth}
			\begin{center}
				\vspace{2cm}
				Master's thesis in\\
				Applied Computer Science\\
				\titlespace
				\mytitle{CoolingGen}
				A parametric 3D-modeling software for turbine blade cooling geometries using NURBS\\
				\titlespace
				\today\\
				\hugespace
				Institute for Numerical and Applied Mathematics at the Georg-August-University Göttingen\\
				\titlespace
				Institute for Propulsion Technology at the German Aerospace Center in Göttingen\\
				\titlespace
				Bachelor's and master's theses at the Center for Computational Sciences at the Georg-August-University Göttingen\\
				\titlespace
				Julian Lüken\\
				\texttt{julian.lueken@dlr.de}\\
			\end{center}
		\end{minipage}
	\end{center}
\end{titlepage}

\restoregeometry
\newgeometry{left=45mm, right=45mm}
\pagestyle{empty}
\pagebreak

\newgeometry{top=210mm}
\noindent
\begin{tabular}{l}
Georg-August-University Göttingen\\
Institute of Computer Science\\
\end{tabular}\\[1em]
\begin{tabular}{ll}
	\Telefon 	&+49 (551) 39-172000\\
	\FAX 		&+49 (551) 39-14403\\
	\Letter 	&\texttt{office@cs.uni-goettingen.de}\\
\end{tabular}\\[1em]
\begin{tabular}{l}
\texttt{www.informatik.uni-goettingen.de}\\
\end{tabular}\\[1em]
\pagebreak

\noindent I hereby declare that this thesis has been written by myself and no other resources than those mentioned have been used.\\[0.7em]
\phantom{H}\includegraphics[height=3em]{../assets/formal/sign.png}\\[0.5em]
Göttingen, \today \hspace{2em}
\pagebreak

\restoregeometry
\begin{abstract}
	\thispagestyle{plain}
	\pagenumbering{roman}
	\setcounter{page}{3}
	\lipsum[1]
\end{abstract}
\pagebreak

\setcounter{page}{4}
\restoregeometry
\newgeometry{left=30mm, right=30mm, top=10mm, bottom=20mm}
\tableofcontents
\pagebreak

\restoregeometry
\pagenumbering{arabic}
\setcounter{page}{1}
\pagestyle{headings}

\chapter{Introduction}
\section{Motivation}
\section{State of the Art}
\section{Problem Statement}

\chapter{Methods}
\section{Bézier Curves}
Bézier curves are named after the french engineer Pierre Bézier, who famously utilized them in the 1960s to design car bodies for the automobile manufacturer Renault. Today, they are used in a wide variety of vector graphics applications (i.e. in font representation on computers). At first glance, the definition of the Bézier curve might seem cumbersome, but given the mathematical foundation and a few graphical representations, it becomes apparent why they are such a powerful tool in computer-aided design.

\subsection{Definition}
\begin{definition}
	The \emph{Bernstein basis polynomials} of degree $n$ on the interval $[t_0,t_1]$ are defined as
	\begin{equation}\label{bernsteinbasisdef}
		b_{n,k,[t_0, t_1]}(t) := \frac{\binom{n}{k} (t_1-t)^{n-k}(t-t_0)^k}{(t_1-t_0)^n,}
	\end{equation}
	for $k \in \{0\dots n\}$.
\end{definition}

\begin{theorem}
	The set of polynomials
		$$\mathcal{P}_n := \{ p : t \mapsto \sum_{k=0}^n c_k t^k, c_k \in \mathbb{R}\}$$
	equipped with the usual operations is a vector space.
\end{theorem}

\begin{theorem}
	The Bernstein basis polynomials of degree $n$ form a basis of $\mathcal{P}_n$.
\end{theorem}

\begin{theorem}[Theorem of Stone-Weierstrass]
	Yet to come.
\end{theorem}

\begin{definition}
	A \emph{Bézier curve} of degree $n$ is a parametric curve $C_{P,[t_0, t_1]}: [t_0, t_1] \rightarrow \mathbb{R}^3$ that has a representation
	\begin{equation}\label{bezierdef}
		C_{P, [t_0, t_1]}(t) = \frac{\sum_{i=0}^n \binom{n}{k} (t_1-t)^{n-k}(t-t_0)^k P_k}{(t_1-t_0)^n}.
	\end{equation}
	We call the elements of the set $P = \{P_1, P_2, \dots, P_n\}$ the \emph{control points} of $C_P$.
\end{definition}

\begin{remark}
	Let $t_0 = 0$ and $t_1 = 1$. Then \ref{bezierdef} simplifies to
	\begin{equation}
		b_{n,k}(t) := b_{n,k,[0,1]}(t) = \binom{n}{k} (1-t)^{n-k}(t)^k.
	\end{equation}
	Also, \ref{bernsteinbasisdef} simplifies to
	\begin{equation}
		C_P(t) := C_{P,[0,1]}(t)= \sum_{i=0}^n \binom{n}{k} (1-t)^{n-k}t^k P_k.
	\end{equation}
	This case is the only case considered in this thesis.
\end{remark}




\subsection{Properties}
\subsection{De Casteljau's Algorithm}

\section{Non-Uniform Rational B-Splines (NURBS)}
\subsection{Definition}
\subsection{Properties}
\subsection{De Boor's Algorithm}

\section{Methods on NURBS Objects}
\subsection{Projection, Translation and Rotation}
\subsection{The Frenet-Serret Apparatus}
\subsection{Finding Intersections}
\subsection{Interpolation}

\section{Jet Engine Design Specifics}
\subsection{Fundamental Terms}
\subsection{The S2M Net}
\subsection{Fillet Creation}

\chapter{Results}
\section{Cooling Geometries And Their Parametrizations}
\subsection{Chambers}
\subsection{Turnarounds}
\subsection{Slots}
\subsection{Film Cooling Holes}
\subsection{Impingement Inserts}
\section{Export for CENTAUR}
\section{Export for Open CASCADE}

\chapter{Discussion}
\section{Future Work}
\section{Conclusion}

\end{document}